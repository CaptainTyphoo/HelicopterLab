\section{Optimal Control of Pitch/Travel and Elevation with and without Feedback}\label{sec:prob4}
In this part of the excersice a constraint on the elevation is added. Therefore the equation describing the dynamics of the elevation $e$ from \eqref{} must be added to the state space representation of the model of the helicopter
\begin{subequations}
	\begin{equation}
	\frac{\d\vec{x}}{\d t}=\vec{A}_c\vec{x}+\vec{B}_c\vec{u}
	\end{equation}
	with
	\begin{align}
	\vec{A}&=\begin{bmatrix}
    0 & 1 & 0 & 0 & 0 & 0\\ 
	0 & 0 & -K_2 & 0 & 0 & 0\\ 
	0 & 0 & 0 & 1 & 0 & 0\\ 
	0 & 0 & -K_1 K_{pp} & -K_1*K_{pd} & 0 & 0 \\
	0 & 0 & 0 & 0 & 0 & 1\\
	0 & 0 & 0 & 0 & -K_3*K_{ep} & -K_3*K_{ed}
	\end{bmatrix}\\
	\vec{B}&=\begin{bmatrix}
	0 & 0\\ 
	0 & 0\\ 
	0 & 0\\ 
	K_1 K_{pp} & 0\\ 
	0 & 0\\ 
	0 & K_3 K_{ep}
	\end{bmatrix}\\
	\vec{x}&=\begin{bmatrix}
	\lambda &
	r&
	p&
	\dot{p}&
	e&
	\dot{e}
	\end{bmatrix}^T\\
	\vec{u}&=\begin{bmatrix}
	p_c & e_c
	\end{bmatrix}^T
	\FullStop
	\end{align}
\end{subequations}
The new input $e_c$ is the stepoint of the elevation. The continuous model is then converted to a time discrete model 
\begin{equation}
\vec{x}_{t+1}=\vec{A}\vec{x}_t+\vec{B}\vec{u}
\end{equation}with the forward Euler method
\begin{subequations}
	\begin{align}
	\vec{A}&=\vec{I}_{6\times6}+\vec{A}_c \Delta t\\
	\vec{B}&=\vec{B}_c\Delta t
	\end{align}
	\label{eq:problem4_disc_model}%
\end{subequations}
with $\Delta t$ being the sampling time. 

The cost function 
\begin{equation}
\phi=\sum_{i=1}^{N}\mybr{\lambda_i-\lambda_f}^2+q_1p_{ci}^2+q_2e_{ci}^2
\end{equation}
is used as a minimization criteria, with the final value for the travel $\lambda_f=0$ and $q_1=1$ and $q_2=2$. The values for $q_1$ and $q_2$ are chosen this way to reduce the oscillations in the opimal trajectory of $p$ and $\dot{p}$ which occur if $q_1=q_2=1$ is used.

The initial value $\vec{x}_0=\begin{bmatrix}\pi & 0 & 0 & 0 & 0 & 0\end{bmatrix}^T$ is used to ensure a travel distance of $\pi$.


As in \cref{sec:prob2} a constraint of $\pm\SI{30}{\degree}$ is used for the pitch $p$ and the setpoint of the pitch $p_c$. Input constraints of $\pm\SI{60}{\degree}$ for $e_c$ are added to avoid a collision between the helicopter and the table on which the helicopter is mounted. Since \eqref{eq:problem4_disc_model} needs to be valid at each time step the equaions are added as equality constraints
\begin{subequations}
	\begin{align}
	\vec{A}_{eq}&=\begin{bmatrix}
	\vec{I} & \vec{0} & \cdots & \cdots & \vec{0} & -\vec{B} & \vec{0} & \cdots &\cdots & \vec{0}\\
	-\vec{A} & \vec{I} & \ddots & & \vdots & \vec{0} & \ddots & \ddots &  & \vdots\\
	\vec{0} & \ddots & \ddots & \ddots & \vdots & \vdots & \ddots & \ddots & \ddots &\vdots \\
	\vdots & \ddots & \ddots & \ddots & \vec{0} & \vdots &  & \ddots & \ddots &\vec{0} \\
	\vec{0} & \cdots & \vec{0} & -\vec{A} & \vec{I} & \vec{0} & \cdots & \cdots & \vec{0} &-\vec{B}\\
	\vec{0} & \cdots & \cdots & \cdots & \vec{I} & \vec{0} & \cdots & \cdots & \cdots & \vec{0}\\
	\end{bmatrix}\\
	\vec{B}_{eq}&=\begin{bmatrix}
	\vec{A}\vec{x}_0\\
	\vec{0}\\
	\vdots\\
	\vec{0}\\
	\vec{x}_f
	\end{bmatrix}\Comma
	\end{align}
\end{subequations}
with $\vec{x}_f=\begin{bmatrix} 0 & 0 & 0 & 0 & 0 & 0\end{bmatrix}^T$ being the final state at $t=N$. A nonlinear constraint 
\begin{equation}
c\mybr{\vec{x}_k}=\alpha \exp\mybr{-\beta\mybr{\lambda_k-\lambda_t}^2}-e_k\leq0 \quad \forall k\in\{1,\ldots,N\}
\label{eq:problem4_nonlinear_constraint}
\end{equation}
with $\alpha=0.2$, $\beta=20$ and $\lambda_t=\frac{2\pi}{3}$, is added. Since \eqref{eq:problem4_nonlinear_constraint} is nonlinear the optimization problem is nonlinear and therefore a nonlinear solver is used. The MATLAB command \verb|fmincon| with three different algorithms is used. The SQP algorithm converges to a solution where the tarjectory of the travel $\lambda$ consists of a single step from $\pi$ to 0, which is unphysical and therefore cannot be used as a reference trajoctory for the helicopter. The active-set method converges to a solution where the input $u_1$ is \SI{83}{\percent} of the time at saturation limit. Although this is only the open loop trajectory it means that when the loop is closed using the LQR the input is still at the saturation limit (assuming that the model is perfect and that there are no disturbances) which means that the control loop is actually open. Because of that the interior-point method is used which results in an trajectory where the input is only \SI{8}{\percent} of the time at the saturation limit. The computation time for calculating the trajectory are \SI{0.33}{\second} for the SQP method, \SI{8.06}{\second} for the active-set method and \SI{53.51}{\second} for the interior-point method, but this is of little concern since the optimization problem doesn't need to be solved online.

The time curve of the helicopter without feedback can be seen in \cref{fig:problem4plots_without_feedback}. As in \cref{sec:prob2} the trajectory of the pitch $p$ is followed, which is due to the pitch control loop which helps to counteract for modeling errors and that a linear model of the nonlinear system is used. The same applies for the elevation control loop which had a reference point of zero in the last two sections and has a trajectory unequal to zero due to the nonlinear constraint \eqref{eq:problem4_nonlinear_constraint} in this section. The reference trajectory of the travel $\lambda$ is not followed satisfactorily. This is the case because the travel $\lambda$ is constrolled in open loop and due to modeling errors the input is not calculated correctly which causes the severe deviations.
\begin{figure}[htbp]
	\centering
	\ProblemFourPlot{../MATLAB/Export/problem4plot_without_feedback.csv}{../MATLAB/Export/problem4plots_opt.csv}
	\caption{problem4plots\_without\_feedback}
	\label{fig:problem4plots_without_feedback}%
\end{figure}

\begin{figure}[htbp]
	\centering
	\ProblemFourPlot{../MATLAB/Export/problem4plot_511111_11.csv}{../MATLAB/Export/problem4plots_opt.csv}
	\caption{problem4plot\_511111\_11}
	\label{fig:problem4plot_511111_11}%
\end{figure}


\begin{figure}[htbp]
	\centering
	\ProblemFourPlot{../MATLAB/Export/problem4_LQR_bothconstraintsN=60.csv}{../MATLAB/Export/problem4plots_LQR_bothconstraintsN=60_opt.csv}
	\caption{LQR\_bothconstraintsN=60}
	\label{fig:LQR_bothconstraintsN=60}%
\end{figure}

\begin{figure}[htbp]
	\centering
	\ProblemFourPlot{../MATLAB/Export/problem4_LQR_onlylambdadotN=60_time_to_complete=553.53s.csv}{../MATLAB/Export/problem4plots_LQR_onlylambdadotN=60_time_to_complete=553.53s_opt.csv}
	\caption{LQR\_onlylambdadotN=60\_time\_to\_complete=553.53s}
	\label{fig:LQR_onlylambdadotN=60_time_to_complete=553.53s}%
\end{figure}

\cleardoublepage


\begin{figure}[htbp]
	\centering
	\ProblemFourPlot{../MATLAB/Export/problem4plot_101010_11.csv}{../MATLAB/Export/problem4plots_opt.csv}
	\caption{problem4plot\_101010\_11}
	\label{fig:problem4plot_101010_11}%
\end{figure}

\begin{figure}[htbp]
	\centering
	\ProblemFourPlot{../MATLAB/Export/problem4plot_111111_11.csv}{../MATLAB/Export/problem4plots_opt.csv}
	\caption{problem4plot\_111111\_11}
	\label{fig:problem4plot_111111_11}%
\end{figure}

\begin{figure}[htbp]
	\centering
	\ProblemFourPlot{../MATLAB/Export/problem4plot_211111_11.csv}{../MATLAB/Export/problem4plots_opt.csv}
	\caption{problem4plot\_211111\_11}
	\label{fig:}%
\end{figure}

\begin{figure}[htbp]
	\centering
	\ProblemFourPlot{../MATLAB/Export/problem4plot_511111_11.csv}{../MATLAB/Export/problem4plots_opt.csv}
	\caption{problem4plot\_511111\_11}
	\label{fig:problem4plot_511111_11}%
\end{figure}

\begin{figure}[htbp]
	\centering
	\ProblemFourPlot{../MATLAB/Export/problem4plot_10011111_11.csv}{../MATLAB/Export/problem4plots_opt.csv}
	\caption{problem4plot\_10011111\_11}
	\label{fig:problem4plot_10011111_11}%
\end{figure}

\begin{figure}[htbp]
	\centering
	\ProblemFourPlot{../MATLAB/Export/problem4plot_21100111_11.csv}{../MATLAB/Export/problem4plots_opt.csv}
	\caption{problem4plot\_21100111\_11}
	\label{fig:problem4plot_21100111_11}%
\end{figure}

\begin{figure}[htbp]
	\centering
	\ProblemFourPlot{../MATLAB/Export/problem4plot_100110111_11.csv}{../MATLAB/Export/problem4plots_opt.csv}
	\caption{problem4plot\_100110111\_11}
	\label{fig:problem4plot_100110111_11}%
\end{figure}

\begin{figure}[htbp]
	\centering
	\ProblemFourPlot{../MATLAB/Export/problem4plot_111111_1001.csv}{../MATLAB/Export/problem4plots_opt.csv}
	\caption{problem4plot\_111111\_1001}
	\label{fig:problem4plot_111111_1001}%
\end{figure}

\begin{figure}[htbp]
	\centering
	\ProblemFourPlot{../MATLAB/Export/problem4plot_111111_1100.csv}{../MATLAB/Export/problem4plots_opt.csv}
	\caption{problem4plot\_111111\_1100}
	\label{fig:problem4plot_111111_1100}%
\end{figure}

\begin{figure}[htbp]
	\centering
	\ProblemFourPlot{../MATLAB/Export/problem4plot_without_feedback.csv}{../MATLAB/Export/problem4plots_opt.csv}
	\caption{problem4plots\_without\_feedback}
	\label{fig:problem4plots_without_feedback}%
\end{figure}

\begin{figure}[htbp]
	\centering
	\ProblemFourPlot{../MATLAB/Export/problem4plot_111111_11_opt_ed_1_lambdad_1.csv}{../MATLAB/Export/problem4plots_opt.csv}
	\caption{problem4plot\_111111\_11\_opt\_ed\_1\_lambdad\_1}
	\label{fig:problem4plot_111111_11_opt_ed_1_lambdad_1}%
\end{figure}

\begin{figure}[htbp]
	\centering
	\ProblemFourPlot{../MATLAB/Export/problem4plot_111111_11_opt_ed_1_pd_1.csv}{../MATLAB/Export/problem4plots_opt.csv}
	\caption{problem4plot\_111111\_11\_opt\_ed\_1\_pd\_1}
	\label{fig:problem4plot_111111_11_opt_ed_1_pd_1}%
\end{figure}





\begin{figure}[htbp]
	\centering
	\ProblemFourPlot{../MATLAB/Export/problem4_NoLQR.csv}{../MATLAB/Export/problem4plots_opt.csv}
	\caption{NoLQR}
	\label{fig:NoLQR}%
\end{figure}

\begin{figure}[htbp]
	\centering
	\ProblemFourPlot{../MATLAB/Export/problem4_NoLQR_bothconstraints.csv}{../MATLAB/Export/problem4plots_bothconstraints_opt.csv}
	\caption{NoLQR\_bothconstraints}
	\label{fig:NoLQR_bothconstraints}%
\end{figure}

\begin{figure}[htbp]
	\centering
	\ProblemFourPlot{../MATLAB/Export/problem4_NoLQR_noAdditionalConstraintsN=60.csv}{../MATLAB/Export/problem4plots_opt.csv}
	\caption{NoLQR\_noAdditionalConstraintsN=60}
	\label{fig:NoLQR_noAdditionalConstraintsN=60}%
\end{figure}

\begin{figure}[htbp]
	\centering
	\ProblemFourPlot{../MATLAB/Export/problem4_NoLQR_onlyedotN=40.csv}{../MATLAB/Export/problem4plots_NoLQR_onlyedotN=40_opt.csv}
	\caption{NoLQR\_onlyedotN=40}
	\label{fig:NoLQR_onlyedotN=40}%
\end{figure}

\begin{figure}[htbp]
	\centering
	\ProblemFourPlot{../MATLAB/Export/problem4_NoLQR_onlylambdadotN=60.csv}{../MATLAB/Export/problem4plots_opt.csv}
	\caption{NoLQR\_onlylambdadotN=60}
	\label{fig:NoLQR_onlylambdadotN=60}%
\end{figure}







\clearpage

\begin{figure}[htbp]
	\centering
	\ProblemFourPlot{../MATLAB/Export/problem4_LQR_bothconstraintsN=60.csv}{../MATLAB/Export/problem4plots_LQR_bothconstraintsN=60_opt.csv}
	\caption{LQR\_bothconstraintsN=60}
	\label{fig:LQR_bothconstraintsN=60}%
\end{figure}

\begin{figure}[htbp]
	\centering
	\ProblemFourPlot{../MATLAB/Export/problem4_LQR_noadditionalN=40_active-set_t=8.0595s.csv}{../MATLAB/Export/problem4plots_LQR_noadditionalN=40_active-set_t=8.0595s_opt.csv}
	\caption{LQR\_noadditionalN=40\_active-set\_t=8.0595s}
	\label{fig:LQR_noadditionalN=40_active-set_t=8.0595s}%
\end{figure}

\begin{figure}[htbp]
	\centering
	\ProblemFourPlot{../MATLAB/Export/problem4_LQR_noadditionalN=40_sqp_t=0.33s.csv}{../MATLAB/Export/problem4plots_LQR_noadditionalN=40_sqp_t=0.33s_opt.csv}
	\caption{LQR\_noadditionalN=40\_sqp\_t=0.33s}
	\label{fig:LQR_noadditionalN=40_sqp_t=0.33s}%
\end{figure}

\begin{figure}[htbp]
	\centering
	\ProblemFourPlot{../MATLAB/Export/problem4_LQR_noadditionalN=60.csv}{../MATLAB/Export/problem4plots_LQR_noadditionalN=60_opt.csv}
	\caption{LQR\_noadditionalN=60}
	\label{fig:LQR_noadditionalN=60}%
\end{figure}

\begin{figure}[htbp]
	\centering
	\ProblemFourPlot{../MATLAB/Export/problem4_LQR_onlyedotN=40.csv}{../MATLAB/Export/problem4plots_LQR_onlyedotN=40_opt.csv}
	\caption{LQR\_onlyedotN=40}
	\label{fig:LQR_onlyedotN=40}%
\end{figure}

\begin{figure}[htbp]
	\centering
	\ProblemFourPlot{../MATLAB/Export/problem4_LQR_onlylambdadotN=60_time_to_complete=553.53s.csv}{../MATLAB/Export/problem4plots_LQR_onlylambdadotN=60_time_to_complete=553.53s_opt.csv}
	\caption{LQR\_onlylambdadotN=60\_time\_to\_complete=553.53s}
	\label{fig:LQR_onlylambdadotN=60_time_to_complete=553.53s}%
\end{figure}

\begin{figure}[htbp]
	\centering
	\ProblemFourPlot{../MATLAB/Export/problem4_LQR_onlylamdadotN=40_sqp_time=0.30s.csv}{../MATLAB/Export/problem4plots_LQR_onlylamdadotN=40_sqp_time=0.30s_opt.csv}
	\caption{LQR\_onlylamdadotN=40\_sqp\_time=0.30s}
	\label{fig:LQR_onlylamdadotN=40_sqp_time=0.30s}%
\end{figure}

\begin{figure}[htbp]
	\centering
	\ProblemFourPlot{../MATLAB/Export/problem4_LQR_onlylamdadotN=60_sqp_time=0.4186s.csv}{../MATLAB/Export/problem4plots_LQR_onlylamdadotN=60_sqp_time=0.4186s_opt.csv}
	\caption{LQR\_onlylamdadotN=60\_sqp\_time=0.4186s}
	\label{fig:LQR_onlylamdadotN=60_sqp_time=0.4186s}%
\end{figure}





\subsection{Results and Discussion}
Each of the four problems should have their own discussion of results. 

