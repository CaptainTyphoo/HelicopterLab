\section{Introduction}
In this lab we are going to use the techniques and theory taught in the course of optimization theory and control. To control a small helicopter model. We will be doing this by computing the optimal set of inputs for reaching a setpoint for the helicopter and give a corresponding input for the helicopter and see if it reaches its intended trajectory.

To compute this we need to derive a non-linear model for the dynamics of the system. Then linearize the model around an equilibrium. This method is widely used in
control theory and have a good preformance if the system operates close to the linearization point. Due to errors could occure from linearization and unmodeled coupling and states we will try out a few different setup to measure the performance of the optimal control computed.

One where we use the optimal input directly without any feedback to see if there is any drift in the system indicating erros in either the linerization and/or the model. Also we will try to compensate for the inaccuracy of the model by using a feedback controller to eliminate drift. In this lab we will be using an linear-quadratic regulator to control the feedback loop. We will also discuss using an MPC instead of an LQR. Additional constraints will also be added to see if they have any inpact in limiting the nonlinear effects on the system.