\section{Introduction}Your introduction should contain an overview of the work you were assigned, as well as a few sentences putting the work into a larger perspective. You should also give a quick description of how the report is organized (as is done below).

You should of course put most of the work into doing good work in the lab and then presenting it in the report. When presenting your work in the report, both content and presentation/layout matters. Since your only way of communicating your good effort in the lab is through writing about it here, the way you write about it is essential. This means that even if you have the very best controller but describe it poorly, you will probably not be rewarded for the good results. A plot showing perfect control is worth very little if it is not accompanied by a clear description of what it represents.

Layout is naturally less important than content, but it still matters. You can think of report writing like selling an apartment; when you present your apartment for potential buyers you will of course clean the apartment and make it good looking. How clean the apartment is does of course not determine its value, but it is still important since it influences the subjective value your buyers will put on the apartment. 

\subsection{Software}
You are of course free to use whatever software you want for report writing. You can also submit a handwritten report, although this is probably not a great idea if your handwriting can be hard to read. 

You can also use Word or a similar word processor. However, it is next to impossible to achieve decent layout with Word. The support for vector graphics (discussed later) is extremely poor, and text tends to look pretty bad (bad support for kerning and ligatures). Furthermore, math is both time consuming and difficult to input, and tends to look very ugly. In general, a report written in Word looks like a draft.

It is strongly recommended to use Latex. Unless you tweak the layout too much, your report will almost certainly look very good. Although it can take a bit of effort to get started, it is also much quicker to use than Word and similar programs. The support for math and vector graphics is also great.

If you are new to Latex, you can have a look at the source for this document to get started. You can also look at the presentation by \citet{Berland2010} (in Norwegian) or consult \citet{Oetiker2011}. Another good reason to learn Latex is that you probably don't want to write your master's thesis in something like Word, doing so would likely be very frustrating. Being reasonably fluent in Latex before you get that far will make your thesis work much smoother.

Some of you are probably fluent in Latex and might plan to write the report using it. Please resist the temptation (if any) to change the fonts, make super fancy headers (they are not necessary for a report like this), change the margins, change the paragraph indentation and/or spacing, and similar things.

A great tool for collaborating on Latex documents is ShareLaTeX at \url{https://www.sharelatex.com/}; if you use this you won't have to install anything on your computer. Texmaker at \url{http://www.xm1math.net/texmaker/} is a good cross-platform editor. Some people like Lyx, which is a Latex editor that behaves a little bit like Word.

\subsection{Other Comments}
If you have problems with Latex, the solution is usually just a few Google searches away.

You can write the report in Norwegian or English. Writing in English is encouraged and is great practice, but entirely optional. \emph{Do not interpret any of the advice or suggestions here as requirements.}

This report is organized as follows: Section~\ref{sec:prob_descr} contains a few remarks on report writing and some random Latex advice. An example of a table can be foundin Section~\ref{sec:prob1}, along with two remarks on report writing. Section~\ref{sec:prob2} contains some advice on using plots from MATLAB. A few suggestions for making illustrations are given in Section~\ref{sec:prob3}; a matrix equation can also be found here. Section~\ref{sec:prob4} has a few comments on references and floats in Latex. The closing discussion and concluding remarks are in Sections~\ref{sec:discussion} and~\ref{sec:conclusion}, respectively. Appendix~\ref{sec:matlab} contains a MATLAB file while Appendix~\ref{sec:simulink} shows an example Simulink diagram. The Bibliography can be found at the end, on page~\pageref{sec:bibliography}.