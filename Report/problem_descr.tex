\section{Problem Description}\label{sec:prob_descr}
\subsection{Lab Setup}

The helicopter, as shown on figure \ref{fig:heli1}, is constructed from two main parts. The basis and the arm. The arm has got on one side two propellers and on the opposite side a counter weight. The arm has got 2 degrees of freedom and can therefore move up and down. The two propellers are attached to the arm by a rotary bound, as shows figure \ref{fig:heli2}. 
The body of the helicopter can also rotate around its axis. Combined with the movement of the arm we observe the \textbf{travel}. The rotation of the propellers around the arm is denoted as the \textbf{pitch}. 	
%\begin{figure}[h]
%\includegraphics[\textwidth]{HeliScatch1.png}
%\caption{Heli}
%\label{fig:heli1}
%\end{figure} 

%\begin{figure}[h]
%\includegraphics[\textwidth]{HeliScatch2.png}
%\caption{Heli}
%\label{fig:heli2}
%\end{figure}
The model is given by the quations \eqref{eq:model}. Equation \eqref{eq:model_se_al_elev} describes the elevation, equation \eqref{eq:model_se_al_pitch} accesses the pitch angle. The speed is the derivation of the of the path as described in equation \eqref{eq:model_se_al_lambda}. The travel accelaration is given by equation \eqref{eq:model_se_al_r}.  
%In this section you should describe the lab setup and discuss the model. If you want, you can copy the source code for the model equations:
%%\begin{gather}
%%	\ddot{e} + K_{3} K_{ed} \dot{e} + K_{3} K_{ep} e = K_{3} K_{ep} e_{c} \label{eq:model_elev} \\
%%	\ddot{p} + K_{1} K_{pd} \dot{p} + K_{1} K_{pp} p = K_{1} K_{pp} p_{c} \label{eq:model_pitch} \\
%%	\dot{\lambda} = r \label{eq:model_lambda} \\
%%	\dot{r} = -K_{2} p \label{eq:model_r} 
%%\end{gather}
%Since these equations belong together, it's a good idea to number them like this:    	
%\begin{subequations}
%\label{eq:model}
%\begin{gather}
%	\ddot{e} + K_{3} K_{ed} \dot{e} + K_{3} K_{ep} e = K_{3} K_{ep} e_{c} \label{eq:model_se_elev} \\
%	\ddot{p} + K_{1} K_{pd} \dot{p} + K_{1} K_{pp} p = K_{1} K_{pp} p_{c} \label{eq:model_se_pitch} \\
%	\dot{\lambda} = r \label{eq:model_se_lambda} \\
%	\dot{r} = -K_{2} p \label{eq:model_se_r} 
%\end{gather}
%\end{subequations}
%You can then both reference individual equations (``the elevation equation \eqref{eq:model_se_elev}'') or reference the entire model (``the linear model \eqref{eq:model}''). Regardless of your choice of software, never hard-code a reference, always use dynamic references. 

%You could also align the equations like this:
\begin{subequations}
\label{eq:model_al}
\begin{align}
	\ddot{e} + K_{3} K_{ed} \dot{e} + K_{3} K_{ep} e &= K_{3} K_{ep} e_{c} \label{eq:model_se_al_elev} \\
	\ddot{p} + K_{1} K_{pd} \dot{p} + K_{1} K_{pp} p &= K_{1} K_{pp} p_{c} \label{eq:model_se_al_pitch} \\
	\dot{\lambda} &= r \label{eq:model_se_al_lambda} \\
	\dot{r} &= -K_{2} p \label{eq:model_se_al_r} 
\end{align}
\end{subequations}
%You can consult \citet{Downes2002} for more about writing math.
These equations were derived from: 
\begin{equation}
	J_{2} \ddot{e} = l_{a} K_{f} V_{s} - T_{g}
	\end{equation}
s.t.
\begin{equation*}
\ddot{e}=K_{3} V_{s}-\frac{T_{g}}{J_{e}},\: K_{3}=\frac{l_{a}K_{f}}{J_{e}}.
\end{equation*}
The model is subsequently discretized into
\begin{equation*}
\Delta x_{i+1}=A\Delta x_{i} + B\Delta u_{i}
\end{equation*} 	
where
\begin{align}
\Delta x &= x -x^* \\
\Delta u &= u -u^*.
\end{align}
We want to minimize the cost function 
\begin{equation}
\phi = \sum_{i=1}^{N}(\lambda_{i}-\lambda_{f})^2+qp_{ci}^{2},\:q\geq 0
\end{equation}
\subsection{Introduction to Simulink /QuaRC}

%If you decide to include a figure, that's great. You can in general copy figures from the textbook, the assignement text, or other places. However: ALWAYS CITE THE SOURCE.